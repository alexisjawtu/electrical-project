% prototype.tex

1.
sla 						= 79 metros cuadrados
grado de electrificaci\'on 	= medio

2. Determinar la corriente de proyecto


PMUs := puntos m\ínimos de utilizaci\'on
		(cantidades m\'inimas de bocas de ilum y de tomacorrientes)
		No se consideran bocas a ninguno de los siguientes:
			* cajas de paso y/o derivaci\'on
			* cajas que contienen exclusivamente elementos de maniobra como los interruptores de efecto
			
Para grado medio:

\textbf{IUG}: 1 boca cada 18 m2 o fracción. Como mínimo una boca.
\textbf{TUG}: 1 boca cada  6 m2 o fracción. Como mínimo dos bocas.
			
El agregado de m\'as bocas o m\'as circuitos no afecta el grado de electrificaci\'on.
No obstante, s\'i impacta en la demanda de potencia máxima simultánea (DPMS), para la
cual se deben tener en cuenta todos los consumos proyectados. (\ref Guía secc. 4)

Comedor: área menor o igual a 

	5.15 x 3.63 = 18.7 (metros cuadrados). 

Calculando por exceso los PMUS
para esta superficie tenemos cubiertos los puntos mínimos requeridos para la superficie real.
Hacemos esto por simplicidad, por la geometría particular del ambiente.

Por la tabla del principio, con dos bocas IUG y 4 bocas TUG estamos cumpliendo con los PMUs para este ambiente.


