% prototype.tex


1) Superficie y grado de electrificación del inmueble

Estamos suponiendo Intensidad de corriente presunta de cortocircuito
en el origen de la instalación no mayor a 10 kAmps.

S.L.A. (aproximado)		= 79 metros cuadrados
grado de electrificación 	= medio

Cómo obtuvimos la S.L.A.:

f: Pulgadas --> Metros
f(x) = m*x

como f(5.25) = 7, entonces
     
     f(5.25) = m*5.25 = 7
    
con lo cual

	m = 7 / 5.25 = 4/3 ~= 1.33
	
Entonces la conversión queda:

			+------------------------+
			| f: Pulgadas --> Metros |
			|			 |
			|         4              |
			| f(x) = --- * x         |
			|         3              |
			+------------------------+

2) y 6) (juntos) Determinar la corriente de proyecto y Determinar los circuitos correspondientes.

PMUs := puntos m\ínimos de utilizaci\'on
		(cantidades m\'inimas de bocas de ilum y de tomacorrientes)
		No se consideran bocas a ninguno de los siguientes:
			* cajas de paso y/o derivaci\'on
			* cajas que contienen exclusivamente elementos de maniobra como los interruptores de efecto
			
Para grado medio:

IUG: 1 boca cada 18 m2 o fracción. Como mínimo una boca.
TUG: 1 boca cada  6 m2 o fracción. Como mínimo dos bocas.
			
El agregado de m\'as bocas o m\'as circuitos no afecta el grado de electrificaci\'on.
No obstante, s\'i impacta en la demanda de potencia máxima simultánea (DPMS), para la
cual se deben tener en cuenta todos los consumos proyectados. (\ref Guía secc. 4)

Comedor: área menor o igual a 

	5.15 x 3.63 = 18.7 (metros cuadrados),
	
y esta es el área que usamos en las tablas adjuntas en el proyecto. 

Calculando por exceso los PMUS para esta superficie tenemos cubiertos los puntos mínimos requeridos para la superficie real.
Hacemos esto por simplicidad, por la geometría particular del ambiente.

Por la tabla del principio, con dos bocas IUG y 4 bocas TUG estamos cumpliendo con los PMUs para este ambiente.

Ponemos el timbre de entrada como una boca IUG más pues en la página 25 de la Guía AEA
podemos leer:

		"...la alimentación a las fuentes de los circuitos MBTF (muy baja tensión
		funcional) puede hacerse desde circuitos IUG, donde cada uno de ellos se 
		contará como una boca de Iluminación de usos Generales, tanto para la 
		potencia como para el número de bocas..."


En los dormitorios 2 y 3 presupuestamos dos bocas IUG porque en cada uno
instalaremos una boca en donde funcionará un artefacto combinado de 
ventilados con luces.


Ambas columnas de IUG y TUG superan las quince bocas con lo cual
dividiremos al conjunto de bocas para tener circuitos que cumplan
con este máximo.


El tablero seccional principal irá instalado en la cocina, en la pared
detrás de la puerta.

Circuitos enumerados con 1: dormitorios (1, 2 y 3), baño, pasillo y galería.
Circuitos enumerados con 2: comedor, cocina, recibidor.

DPMS:

Según la Tabla 770.8.I de la reglamentación de AEA y las tablas de la página
22 de la Guía AEA, tenemos lo siguiente.

Circuitos IUG:

Valor de potencia estimado (si no se conoce la carga real): 60 VA por boca
corregido por el factor de simultaneidad 2/3.

Intensidad nominal de ITM de protección <= 16A

Sección de los cables de fase y neutro >= 1.5 mm2

Circuitos TUG:

Valor de potencia estimado (si no se conoce la carga real): 2200 VA por circuito.

Intensidad nominal de ITM de protección <= 20A

Sección de los cables de fase y neutro >= 1.5 mm2

Entonces en nuestro proyecto determinamos:

Circuito IUG-1 (9 bocas): 9 x 60 x 2/3 VA = 360 VA 
Circuito IUG-2 (9 bocas): 9 x 60 x 2/3 VA = 360 VA
Circuito TUG-1:                            2200 VA
Circuito TUG-2:                            2200 VA
Circuito TUG-Acondicionador de Aire:       2200 VA  
--------------------------------------------------
DPMS resultante                            7320 VA

Cálculo de corriente de uso (IB) para bocas IUG:  360/220 Amps =  1.64 Amps
Cálculo de corriente de uso (IB) para bocas TUG: 2200/220 Amps = 10    Amps

3) Determinar el tipo de canalización.

Tipo de canalización embutida de p.v.c. corrugado con accesorios del mismo sistema de 
material aislante.

Vamos a instalar una cañería por circuito para no quedar restringidos a quince bocas según 
AEA 770.10.3.8.2 inciso d, en donde se aclara que la suma de las cantidades de bocas de 
los circuitos que compartan conductos no puede exceder ese número. 

4) Determinar el calibre de los conductores.

Una vez dererminada la DPMS, estamos en condiciones de elegir los cables de acuerdo
con su forma de instalación y factores de agrupamiento.

El proyecto no tiene circuitos de uso especial ni cargas específicas con lo que 
la carga total es exactamente la DPMS para el grado de electrificación.

Para la Iz de los cables estamos tomando en cuenta que los circuitos recorren cañerías
embutidas y lo forman cables según norma IRAM NM 247-3.

5) Determinar el cuadro eléctrico; su conformación.
 
Tablero Principal (TP)
*Instalado en la acometida del medidor. 

Conformación:
*Caja estanco.
*Interruptor termo-magnético con intensidad nominal de X amperios.
*La barra de tierra, que se usará como punto de ingreso del conductor de P. a T. a la instalación, con
         -pletina con perforaciones roscadas para tornillos
	 -seis o más bornes de P. a T. montados sobre riel DIN.

Tablero Seccional Principal (T.S.P.)
*Se instalará detrás de una de las puertas de la cocina como se indica en los planos 

Conformación:
*1 Interruptor diferencial
*1 Interruptor termo-magnético (TUG-1)
*1 Interruptor termo-magnético (TUG-2)
*1 Interruptor termo-magnético (TUG-A.A.)
*1 Interruptor termo-magnético (IUG-1)
*1 Interruptor termo-magnético (IUG-2)

6) Determinar los circuitos correspondientes

Ver el punto 2).

7) Cuadro de referencia de la simbología eléctrica (ver el diagrama unifilar en el reverso del plano)

8) Puesta a tierra
* Una jabalina cilíndrica de hierro con baño electrolítico de cobre IRAM 2309 o 2310.
* Un tomacable de bronce
* Una caja plástica de inspección con tapa removible
* Una barra de tierra para el tablero principal
* Cable con vaina verde y amarilla con superficie seccional de 4 milímetros cuadrados para conectar
el tomacable de la jabalina con la barra del tablero principal.

9) Cálculo de materiales.

Presupuesto de mano de obra 
===========================

Última fecha de validez del importe:
====================================

